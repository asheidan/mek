\documentclass[a4]{article}
\usepackage{hyperref}
\usepackage[swedish]{babel,varioref}
\usepackage[utf8]{inputenc}
\usepackage{graphicx}

\begin{document}
{
\Large
Solsystemet\\
Klassisk Mekanik\\
~\\
Benjamin Nauck, c05ben@cs.umu.se\\
Emil Eriksson, c07een@cs.umu.se\\
~\\
%\today{}
}
\clearpage

\section{Introduktion}
I följande laboration kommer vi att gå igenom svar på uppgifter vi fått och
diskussion om dessa.
Laborationen gick ut på att implementera och analysera n-kropps-problem med
hjälp av Velocity-Verlet-simulering.

\section{Simulering av satellit}
Satellitens omloppsbana går att finna i Figur~\vref{fig:satellite}.
För att satellitens omloppsbana ska bli stabil behöver $\Delta t$ vara kring
$100s$.
%Se Figur~\vref{fig:satellite:findstableorbit}
\begin{figure}
\begin{center}
	\includegraphics[width=\textwidth]{../uppg1_orbit.png}
\end{center}
\caption{Satellitens omloppsbana.}
\label{fig:satellite}
\end{figure}

%\begin{figure}
%\begin{center}
%	\includegraphics[width=\textwidth]{}
%\end{center}
%\caption{Satellitens bana när $\Delta t$ förändras.}
%\label{fig:satellite:findstableorbit}
%\end{figure}

I Figur~\vref{fig:satellite:energy} går det att se hur energin i systemet ändras
över tiden.
I Figur~\vref{fig:satellite:linearmomentum} går det att se hur rörelsemängden i
systemet ändras över tiden.

\begin{figure}
\begin{center}
	\includegraphics[width=\textwidth]{../uppg1_energy.png}
\end{center}
\caption{Energin för satelliten i omloppsbana.}
\label{fig:satellite:energy}
\end{figure}

\begin{figure}
\begin{center}
	\includegraphics[width=\textwidth]{../uppg1_momentum.png}
\end{center}
\caption{Rörelsemängden för satelliten.}
\label{fig:satellite:linearmomentum}
\end{figure}


\section{Omloppstiden}
	\subsection{Beräkning av omloppstid}
För att beräkna omloppstiden itererar vi igenom positionerna och ser när
positionen går in i första kvadranten.

Omloppstiden för satelliten är beräknad till: $37s$.

Om initialhastigheten ökas till runt $1.4$ lämnar satelliten sin omloppsbana.
Se Figur~\vref{fig:orbit:mutlipleorbits}.
I Figur~\vref{fig:orbit:timeforvelocity} kan vi se att omloppstiden ökar exponensiellt
med initialhastigheten.

\begin{figure}
\begin{center}
	\includegraphics[width=\textwidth]{../uppg2_orbits.png}
\end{center}
\caption{
Omloppsbanor då initialhastigheten förändras hos satelliten.
}
\label{fig:orbit:mutlipleorbits}
\end{figure}

\begin{figure}
\begin{center}
	\includegraphics[width=\textwidth]{../uppg2_energy.png}
\end{center}
\caption{
Som vi kan se så ökar omloppstiden exponensiellt med initialhastigheten.
Som förväntat ökar energin kvadratiskt.
}
\label{fig:orbit:timeforvelocity}
\end{figure}

	\subsection{ISS}
För att undersöka om våran simuleringsmodell fungerar testar vi att simulera
rymdstationen ISS omloppsbana kring jorden med verklig data.

Den data vi fört in i system återfinns i Tabell~\vref{iss:data}.
\begin{table}
\begin{center}
\begin{tabular}{c|c}
	konstant & värde \\
	\hline
	$G$       &  $6.67384 \cdot 10^{-11}$ \\
	$m_J$     &  $5.972 \cdot 10^{24}$ \\
	$m_{\mathrm{ISS}}$ &  $4.5 \cdot 10^5$ \\
	$v_0$     &  $7.7066 \cdot 10^3$ \\
	$h$       &  $4.12 \cdot 10^5$ \\
	$r_J$     &  $6.371 \cdot 10^6$
\end{tabular}
\caption{
Tabell över data använt i simulering av ISS omloppsbana
kringjorden.[wikipedia]}
\label{iss:data}
\end{center}
\end{table}

Den simulerade omloppstiden för ISS är beräknad till $5651s$ vilket tycks
stämma ganska bra med det riktiga värdet som ligger på $5570s$ [wikipedia].

\begin{figure}
\begin{center}
	\includegraphics[width=\textwidth]{../uppg2_iss.png}
\end{center}
\caption{
	ISS simulerade omloppsbana kring jorden för ett antal värden på $\Delta t$
}
\end{figure}

\section{Tvåkroppsproblemet}
Denna uppgift är överhoppad då den är så pass lik nästkommande uppgift.

\section{Solsystemet}
För att testa om simuleringen fungerar för $n$ stycken kroppar så testar vi att
simulera solsystemets inre kroppar; Solen, Merkurius, Venus, Jorden samt Mars.
Data för dessa kroppar återfinns i Tabell~\vref{solarsystem:data}.

\begin{table}
\begin{center}
\begin{tabular}{c|c|c|c}
	Planet    & Massa (relativt jordens) & Avstånd till Solen (mätt i $AU$) & hastighet\\
	\hline
	Merkurius & $5.5 \cdot 10^{-2}$  & $4 \cdot 10^{-1}$  &  $4.787 \cdot 10^4$ \\
	Venus     & $8.15 \cdot 10^{-1}$ & $7 \cdot 10^{-1}$  &  $3.502 \cdot 10^4$ \\ 
	Jorden    & $1$                  & $1$                &  $2.978 \cdot 10^4$ \\
	Mars      & $1.07 \cdot 10^{-1}$ & $1.5$              &  $2.4077 \cdot 10^4$
\end{tabular}
\caption{
	Tabell över de inre planeternas och solens massor, planeternas avstånd till
	solen relativt jordens samt dess hastighet.[wikipedia]
}
\label{solarsystem:data}
\end{center}
\end{table}
	

		\subsection{Stabil bana}
Som nämnts tidigare är ett problem med tidsdiskretiserad simulering, att
resultatet ändras om man justerar tiden mellan varje iteration.
För att en simulering ska bli stabil måste ett tillräckligt lågt $\Delta t$
väljas.
Som går att se i Figur~\vref{fig:solarsystem:large:dt} så kan man få oönskat
resultat detta värde väljs för högt, men om man istället väljer ett
tillräckligt lågt tal så stabiliseras omloppsbanorna, se
Figur~\vref{fig:solarsystem:small:enough:dt}.

%		\subsection{Planetära omloppsbanor} % svarat på ovan

		\subsection{Validering av simulering}
Vidare går det att undersöka om energin och rörelsemängden
%samt rörelse\-mängds\-momentet
bevaras genom simuleringen för att validera
simuleringsmodellen.

I Figur~\vref{fig:solarsystem:kinetic} kan vi se den kinetiska energin för de olika kropparna i solsystemet.
Tillsammans med Figur~\vref{fig:solarsystem:potential} som visar den potentiella energin får man 
Figur~\vref{fig:solarsystem:energy}.
Den totala energin domineras totalt av den potentiella energin och den potentiella energin domineras av bidraget från Solen.
\begin{figure}
\begin{center}
	\includegraphics[width=\textwidth]{../uppg4_orbit_bad}
\end{center}
\caption{Simulerade omloppsbanor med ett för stort värde på $\Delta t$}
\label{fig:solarsystem:large:dt}
\end{figure}

\begin{figure}
\begin{center}
	\includegraphics[width=\textwidth]{../uppg4_orbit}
\end{center}
\caption{Simulering av omloppsbanor i solsystemet}
\label{fig:solarsystem:small:enough:dt}
\end{figure}

\begin{figure}
\begin{center}
	\includegraphics[width=\textwidth]{../uppg4_potential}
\end{center}
\caption{Potentiella energin under simuleringen i Figur~\vref{fig:solarsystem:small:enough:dt}.}
\label{fig:solarsystem:potential}
\end{figure}

\begin{figure}
\begin{center}
	\includegraphics[width=\textwidth]{../uppg4_kinetic}
\end{center}
\caption{Rörelseenergin under simuleringen i Figur~\vref{fig:solarsystem:small:enough:dt}.}
\label{fig:solarsystem:kinetic}
\end{figure}

\begin{figure}
\begin{center}
	\includegraphics[width=\textwidth]{../uppg4_total_energy}
\end{center}
\caption{Totala energin under simuleringen i Figur~\vref{fig:solarsystem:small:enough:dt}.}
\label{fig:solarsystem:energy}
\end{figure}

I Figur~\vref{fig:solarsystem:linearmomentum} kan vi se rörelsemängden i systemet.
Diagrammet visar absolutbeloppet av det totala rörelsemomentet.
%Anledningen till att det har ett så ``stort'' värde trots att det borde vara 0 kan vara att det är ett avrundningsfel.
\begin{figure}
\begin{center}
	\includegraphics[width=\textwidth]{../uppg4_momentum}
\end{center}
\caption{Rörelsemängden under simuleringen i Figur~\vref{fig:solarsystem:small:enough:dt}.}
\label{fig:solarsystem:linearmomentum}
\end{figure}

%Och slutligen i Figur~\vref{fig:solarsystem:energy} kan vi se TODO: add what we can read.

		\subsection{Omloppstider}
Mycket går att analysera genom att undersöka energibevarande och så vidare, men
simulerar man verkligheten bör man även undersöka så att simuleringen verkligen
ter sig som verkligheten.
Av denna anledning har vi undersökt tiden för planeternas omloppsbanor i
Tabell~\vref{table:solarsystem:orbit:reality:check}.
Som vi kan se så stämmer inte våra simulerade tider speciellt bra med
verkligheten.
Den som kan hitta anledningen blir belönad med bulle och tillhörande kaffe.

\begin{table}
\begin{center}
\begin{tabular}{c|c|c}
	Planet    & Simulerad omloppstid & Reell omloppstid \\
	\hline
	Merkurius & $6.046 \cdot 10^6$ &  $7.60053 \cdot 10^7$ \\
	Venus     & $7.470 \cdot 10^5$ &  $1.94139 \cdot 10^7$ \\ 
	Jorden    & $1.570 \cdot 10^7$ &  $3.15581 \cdot 10^7$ \\
	Mars      & $3.498 \cdot 10^7$ &  $5.93542 \cdot 10^7$
\end{tabular}
\caption{
	Tabell över de inre planeternas omloppstider.[wikipedia]
}
\label{table:solarsystem:orbit:reality:check}
\end{center}
\end{table}

\section{Källkod}
	Matlabkoden går att återfinna på:\\
	\url{https://github.com/asheidan/mek} samt
	\url{https://github.com/hyarion/mek}
	

\end{document}
